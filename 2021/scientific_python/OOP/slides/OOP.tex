\documentclass{beamer}
\usetheme{ensam}
\usepackage{pgfplots}
\usepackage{subcaption}
\usepackage{acronym}
\usepackage{csvsimple}
\usepackage{tikz}
\usetikzlibrary{calc}
\usepackage{amsmath}
\usepackage {algorithmic}
\usepackage{algorithm}
\usepackage{eqparbox}
\usepackage[font=scriptsize]{caption}
\usetikzlibrary{bayesnet,positioning,calc}
\tikzstyle{obs} = [latent,fill=lightBlue]
\tikzstyle{default}=[draw=sexyRed,thick,rounded corners,text width=0.5in,font=\scriptsize,align=center]
\usepgfplotslibrary{colorbrewer}
\definecolor{ForestGreen}{RGB}{34,139,34}
\newcommand{\comment}[1]{\textcolor{ForestGreen}{#1}}
%algorithmic comment
\renewcommand\algorithmiccomment[1]{%
  \hfill\comment{\#\scriptsize\eqparbox{COMMENT}{#1}}%
}
\renewcommand{\algorithmicrequire}{\textbf{Input:}}
\renewcommand{\algorithmicensure}{\textbf{Output:}}
\title{Introduction to OOP with Python}
\author{\underline{Formation Python}}
\institute{\small  ENSAF-UEMF} 

%tikz bayesian theme
\usetikzlibrary{bayesnet,positioning,calc}
\tikzstyle{obs} = [latent,fill=lightBlue]
\tikzstyle{default}=[draw=sexyRed,thick,rounded corners,text width=0.5in,font=\scriptsize,align=center]
\DeclareMathOperator{\argmin}{argmin}

\pgfplotsset{every tick label/.append style={font=\tiny}}


\begin{document}
\maketitle

%{{{ Table of contents
\begin{frame}
\tableofcontents
\end{frame}
%}}}
% Introduction {{{ %
\section{Why OOP}
\begin{frame}[t]
  \frametitle{Why OOP}
  \begin{block}{Problème simple}
    Une banque possède un fichier contenant les informations de ces clients.
    Votre tache, si vous l'accepter, consiste a lire ces clients puis les
    afficher selon un ordre croissant des noms.
  \end{block}

  \begin{figure}[htpb]
  \begin{center}
  \begin{tikzpicture}[scale=1, transform shape]
    \node[rounded corners, thick, draw, sexyRed] (proc){\small Approche  procédurale} ;
  \end{tikzpicture}
  \end{center}
  \end{figure}
  
\end{frame}


% Soluiton procedurale {{{ %
% }}} Soluiton procedurale %
% }}} Introduction %


\end{document}
